\part{Test}


\chapter{Overview}





\chapter{Server}




\section{PHP}


从PHP 5.4.0起,CLI SAPI提供了一个内置的Web服务器来用于本地开发使用,不可用于生产环境。

默认情况下,URI请求会被发送到PHP所在的的工作目录(Working Directory)进行处理,除非使用\texttt{-t}参数来自定义不同的目录。

如果请求未指定执行哪个PHP文件,则默认执行目录内的index.php 或者 index.html。如果这两个文件都不存在,服务器会返回404错误。


在命令行启动内置的Web Server时,如果指定了一个PHP文件,则这个文件会作为一个“路由”脚本,意味着每次请求都会先执行这个脚本。如果这个脚本返回 FALSE ,那么直接返回请求的文件(例如请求静态文件不作任何处理),否则会把输出返回到浏览器。

\begin{lstlisting}[language=PHP]
$ cd ~/public_html
$ php -S localhost:10086
PHP 5.6.12 Development Server started at Sun Oct 25 21:51:18 2015
Listening on http://localhost:10086
Document root is /home/test/public_html
Press Ctrl-C to quit.
\end{lstlisting}


如果需要在启动时指定Web根目录,可以执行:

\begin{lstlisting}[language=PHP]
$ cd ~/public_html
$ php -S localhost:8000 -t foo/
\end{lstlisting}


下面的示例说明如何使用路由(router)脚本,请求图片直接显示图片,请求HTML则显示“Welcome to PHP”。

\begin{lstlisting}[language=PHP]
$ cd ~/public_html
$ vim router.php
<?php
// router.php
if (preg_match('/\.(?:png|jpg|jpeg|gif)$/', $_SERVER["REQUEST_URI"]))
    return false;    // 直接返回请求的文件
else { 
    echo "<p>Welcome to PHP</p>";
}
?>
$ $ php -S localhost:8000 router.php
\end{lstlisting}

PHP内置的Web服务器能识别一些标准的MIME类型资源,例如.css, .gif, .htm, .html, .jpe, .jpeg, .jpg, .js, .png, .svg和.txt等。

对于不支持的文件类型,可以进行相应的判断和处理:



通过程序判断是否是在使用内置Web服务器,从而调整同一个PHP路由器脚本在内置Web服务器中和在生产服务器中的不同行为:


\begin{lstlisting}[language=PHP]
$ cd ~/public_html
$ vim router.php
<?php
// router.php
if (php_sapi_name() == 'cli-server') {
    /* route static assets and return false */
}
    /* go on with normal index.php operations */
?>
$ php -S localhost:10086 router.php
\end{lstlisting}




\begin{lstlisting}[language=PHP]
$ cd ~/public_html
$ vim router.php
<?php
// router.php
$path = pathinfo($_SERVER["SCRIPT_FILENAME"]);
if($path["extension"]=="ogg"){
	header("Content-Type: video/ogg");
	readfile($_SERVER["SCRIPT_FILENAME"]);
}else{
	return FALSE;
}
?>
$ php -S localhost:10086 router.php
\end{lstlisting}

如果需要开启服务器的远程访问支持,可以执行:

\begin{lstlisting}[language=PHP]
$ php -S 0.0.0.0:10086
\end{lstlisting}




\begin{lstlisting}[language=PHP]

\end{lstlisting}



\begin{lstlisting}[language=PHP]

\end{lstlisting}




\begin{lstlisting}[language=PHP]

\end{lstlisting}




\begin{lstlisting}[language=PHP]

\end{lstlisting}




\begin{lstlisting}[language=PHP]

\end{lstlisting}




\begin{lstlisting}[language=PHP]

\end{lstlisting}





\begin{lstlisting}[language=PHP]

\end{lstlisting}











\section{Python}


Python自带的Web服务器SimpleHTTPServer可以使用下面的命令来启动(端口号默认为8000):



\begin{lstlisting}[language=PHP]
python -m Web服务器模块 [端口号,默认8000]
\end{lstlisting}


这里的“Web服务器模块”有如下3种:

\begin{compactitem}
\item BaseHTTPServer: 提供基本的Web服务和处理器类,分别是HTTPServer和BaseHTTPRequestHandler。
\item SimpleHTTPServer: 包含执行GET和HEAD请求的SimpleHTTPRequestHandler类。
\item CGIHTTPServer: 包含处理POST请求和执行CGIHTTPRequestHandler类。
\end{compactitem}

\begin{lstlisting}[language=PHP]
$ cd ~/public_html
$ python -m SimpleHTTPServer 10086
\end{lstlisting}

\begin{lstlisting}[language=PHP]

\end{lstlisting}




\begin{lstlisting}[language=PHP]

\end{lstlisting}




\begin{lstlisting}[language=PHP]

\end{lstlisting}




\begin{lstlisting}[language=PHP]

\end{lstlisting}





\begin{lstlisting}[language=PHP]

\end{lstlisting}




\begin{lstlisting}[language=PHP]

\end{lstlisting}







\begin{lstlisting}[language=PHP]

\end{lstlisting}




\begin{lstlisting}[language=PHP]

\end{lstlisting}




\begin{lstlisting}[language=PHP]

\end{lstlisting}




\begin{lstlisting}[language=PHP]

\end{lstlisting}




\begin{lstlisting}[language=PHP]

\end{lstlisting}





\begin{lstlisting}[language=PHP]

\end{lstlisting}




\begin{lstlisting}[language=PHP]

\end{lstlisting}







\begin{lstlisting}[language=PHP]

\end{lstlisting}




\begin{lstlisting}[language=PHP]

\end{lstlisting}




\begin{lstlisting}[language=PHP]

\end{lstlisting}




\begin{lstlisting}[language=PHP]

\end{lstlisting}




\begin{lstlisting}[language=PHP]

\end{lstlisting}





\begin{lstlisting}[language=PHP]

\end{lstlisting}




\begin{lstlisting}[language=PHP]

\end{lstlisting}







\begin{lstlisting}[language=PHP]

\end{lstlisting}




\begin{lstlisting}[language=PHP]

\end{lstlisting}




\begin{lstlisting}[language=PHP]

\end{lstlisting}




\begin{lstlisting}[language=PHP]

\end{lstlisting}




\begin{lstlisting}[language=PHP]

\end{lstlisting}





\begin{lstlisting}[language=PHP]

\end{lstlisting}




\begin{lstlisting}[language=PHP]

\end{lstlisting}







\begin{lstlisting}[language=PHP]

\end{lstlisting}




\begin{lstlisting}[language=PHP]

\end{lstlisting}




\begin{lstlisting}[language=PHP]

\end{lstlisting}




\begin{lstlisting}[language=PHP]

\end{lstlisting}




\begin{lstlisting}[language=PHP]

\end{lstlisting}





\begin{lstlisting}[language=PHP]

\end{lstlisting}




\begin{lstlisting}[language=PHP]

\end{lstlisting}







