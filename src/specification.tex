\part{Specification}



\chapter{Overview}




\chapter{Directory}


<<<<<<< HEAD
项目目录结构规范\footnote{directory\_spec}的设计目标是项目开发的目录结构保持一致,使容易理解并方便构建与管理。
=======
项目目录结构规范\footnote{directory\_spec}的设计目标是项目开发的目录结构保持一致,使容易理解并方便构建与管理。
>>>>>>> d0778385719591c9fc4f1738f7b1b023363dde44


\begin{compactitem}
\item 项目包含但不限于业务项目和包项目。
\item \texttt{\$\{root\}}表示项目的根目录。
\end{compactitem}


资源分成两大类,分别是源代码资源和内容资源。

\begin{compactenum}
\item 源代码资源:开发者编写的源代码,包括js、html、css、template等。
\item 内容资源:希望作为内容提供给访问者的资源,包括图片、字体、flash、pdf等。
\end{compactenum}

\section{Naming}


目录命名原则要求简洁,不允许使用复数。

习惯性缩写的单词 必须(MUST) 采用容易理解的缩写。例如,源代码目录使用src,不使用source。

\begin{compactitem}
\item img(图片),不允许(MUST NOT) 使用image、images、imgs等。
\item js(javascript脚本),不允许(MUST NOT) 使用script、scripts等。
\item css(样式表),不允许(MUST NOT) 使用style、styles等。
\item swf(flash),不允许(MUST NOT) 使用flash等。
\item src(源文件目录),不允许(MUST NOT) 使用source等。
\item dep(引入的第三方依赖包目录),不允许(MUST NOT) 使用lib、library、dependency等。
\end{compactitem}

不允许(MUST NOT) 使用复数形式。例如,imgs、docs是不被允许的。

\section{Structure}

在\$\{root\}下,目录结构必须(MUST) 按照职能进行划分, 不允许(MUST NOT) 将资源类型或业务逻辑划分的目录直接置于\$\{root\}下。

常用的目录有src、doc、dep、test等。

\begin{lstlisting}[language=bash]
${root}/
    src/
    test/
    doc/
    dep/
    ...
\end{lstlisting}


业务项目的\$\{root\}目录结构划分遵循\$\{root\}目录结构划分。


从程序结构的角度来看,必须合理设计项目的目录结构。例如,支持文件和引用文件不能保存在公众可以公开访问的目录中,用户信息必须存储在数据库中。

在项目开发过程中的第一步就是设计良好的项目目录结构。例如,下面是一个PHP项目目录结构的初级设计。

\begin{lstlisting}[language=bash]
${root}/
    pub/
    lib/
    tpl/
\end{lstlisting}


\begin{compactitem}
\item pub/保存所有可以公开访问的页面;
\item lib/保存可以被其他文件调用的引用文件;
\item tpl/保存页面显示文件。
\end{compactitem}

在实际网站中,Web服务器应该只允许访客访问pub/目录下的文件(例如样式表、脚本和图片等),其他文件必须保存在公共目录之外的目录中。


\begin{lstlisting}[language=bash]
${root}/
    pub/
        img/
        css/
        js/
    lib/
    tpl/
\end{lstlisting}


Web服务器的根目录是一个特殊目录,应该通过httpd.conf或URL重写等指向指定的主页。


\begin{lstlisting}[language=bash]
# cat /etc/httpd/conf/httpd.conf
#
# DirectoryIndex: sets the file that Apache will serve if a directory
# is requested.
#
<IfModule dir_module>
    DirectoryIndex index.html
</IfModule>
\end{lstlisting}

除此之外,还需要创建其他的目录来提供与项目开发过程相关的资源,并且保持代码结构的清晰性。


\begin{lstlisting}[language=bash]
${root}/
    pub/
        img/
        css/
        js/
    lib/
    tpl/
    sql/
    doc/
    test/
\end{lstlisting}


\begin{compactitem}
\item sql/保存MySQL文件等;
\item doc/保存帮助文档和开发笔记等;
\item test/保存发烟测试和单元测试代码等。
\end{compactitem}

最后也是最基础的,只要页面需要被用户访问,那么它就应该被放置到可以公开访问的目录中,至于其执行过程则由Web服务器和应用程序本身来负责。


\begin{lstlisting}[language=bash]

\end{lstlisting}




\begin{lstlisting}[language=bash]

\end{lstlisting}




\begin{lstlisting}[language=bash]

\end{lstlisting}





\section{Code}

业务项目可以(SHOULD) 为项目起一个代号名称,代号名称 必须(MUST) 为一个单词,不宜过长。例如,北斗的项目代号为triones,哥伦布的项目代号为clb。

项目代号应该有利于区分不同项目,为未来项目之间的重用留下扩展的后路。


在项目开发时,通常会使用如下加载器配置,将项目代号指向src。


\begin{lstlisting}[language=bash]
{
    baseUrl: '${docroot}',
    paths: {
        'triones': 'src'
    }
}
\end{lstlisting}

业务项目的src目录内,绝大多数情况 应当(SHOULD) 根据业务逻辑划分目录结构。划分出的子目录(比如例子中的biz1)称为业务目录。


src下 必须(MUST) 只包含业务目录与common目录。业务公共资源 必须(MUST) 命名为common。common目录作为业务公共资源的目录,也视如业务目录。




\begin{lstlisting}[language=bash]
${root}/
    src/
        common/
        biz1/
            subbiz1/
            subbiz2/
        biz2/
\end{lstlisting}

较小规模的业务项目(如投放端),src目录允许视如业务目录,直接按照业务目录划分原则划分目录结构。


\begin{lstlisting}[language=bash]
${root}/
    src/
        foo.js
\end{lstlisting}


\section{Database}


在设计项目目录结构的同时也需要根据需求来设计数据库结构。

项目(例如网站)需要从用户处收集的信息和项目本身所能提供的服务相关,而且这些信息又会反过来影响数据库设计中的表结构等。




\section{Principle}

\begin{compactenum}
\item JS资源:不允许(MUST NOT) 按资源类型划分目录, 必须(MUST) 按业务逻辑划分目录。JS资源应直接置于业务目录下。即:业务目录下不允许出现js目录。
\item 除JS资源外的源文件资源,当资源数量较多时,为方便管理, 允许(SHOULD) 按资源类型划分目录。即:业务目录下允许出现css、tpl目录。
\item 内容资源:允许(SHOULD) 按资源类型划分目录。即:业务目录下允许出现img、swf、font目录。
\item 业务目录中,如果文件太多不好管理,需要划分子目录时,也 必须(MUST) 继续遵守根据业务逻辑划分的原则,划分子业务。如:下面例子中的subbiz1。
\end{compactenum}


\subsection{Biz}


通常,对于一个业务目录, 鼓励(SHOULD) 将业务相关的源文件资源都直接置于业务目录下。


\begin{lstlisting}[language=bash]
biz1/
    img/
        add_button.png
    add.js
    add.tpl.html
    add.css
\end{lstlisting}

业务目录下源文件资源数量较多时,我们第一直觉应该是:是否业务划分不够细?是否应该划分子业务,建立子业务目录?



\begin{lstlisting}[language=bash]
biz2/
    subbiz1/
        list.js
        list.tpl.html
        list.css
    subbiz2/
\end{lstlisting}

遇到确实是一个业务整体,无法划分子业务时, 允许(MAY) 将非JS资源按资源类型划分目录进行管理。


\begin{lstlisting}[language=bash]
biz1/
    css/
        add.css
        edit.css
        remove.css
        img/
            add_button.png
    tpl/
        add.html
        edit.html
        remove.html
    add.js
    edit.js
    remove.js
\end{lstlisting}

业务项目目录划分示例


\begin{lstlisting}[language=bash]
${root}/
    src/
        common/
            img/
                sprites.png
                logo.png
            conf.js
            layout.css
        biz1/
            img/
                add_button.png
            add.js
            add.tpl.html
            add.less
        biz2/
            subbiz1/
                list.js
                list.tpl.html
                list.css
            subbiz2/
    dep/
        er/
            src/
            test/
        esui/
            src/
            test/
    test/
    doc/
    index.html
    main.html
    ......
\end{lstlisting}


common目录为业务公共目录,用于存放业务项目的业务公共文件。所以,根据业务逻辑划分目录结构时,业务逻辑命名 不允许(MUST NOT) 为common。

biz下面不能有资源类型目录,否则如果在biz下继续划分资源目录,代码的结构可能如下:



\begin{lstlisting}[language=bash]
${root}/
    src/
        biz1/
            js/
                list.js
\end{lstlisting}

当我们需要使用list.js的时候,必须写如下的代码:\texttt{require("../biz1/js/list")},但是从逻辑上说,更合理的写法应该是\texttt{require("../biz1/list")},因此不推荐在biz下面对源代码资源划分目录。


\subsection{Package}


包项目的\$\{root\}目录结构划分遵循\$\{root\}目录结构划分。

包是实现某个独立功能,有复用价值的代码集。按照通常的理解,一个包项目不应该特别复杂,所以包可视如一个不太复杂的业务,其src下的划分原则与业务项目的业务目录划分原则保持一致。




\begin{lstlisting}[language=bash]
${root}/
    src/
        css/
            img/
                sprites.png
            table.css
            button.css
            select.css
        main.js
        Control.js
        InputControl.js
        Button.js
        Table.js
        Select.js
    test/
    doc/
    package.json
    ...
\end{lstlisting}


直接置于\$\{root\}下的目录称作一级目录。一级目录 必须(MUST) 具有某种职能属性。

除了下面列举的一些常见目录之外,\$\{root\}下也可以放置一些跟项目发布相关的文件,例如build.sh,build.xml,Makefile,Gruntfile等。

\begin{compactitem}
\item src

src目录用于存放开发时源文件,发布时 必须(MUST) 被删除。

\item dep

dep目录用于存放项目引入依赖的第三方包。该目录下的内容通过平台工具管理,项目开发人员 不允许(MUST NOT) 更改dep目录下第三方包的任何内容。

当项目需要修改引入的第三方代码时,第三方包应将源码直接置于\$\{root\}/src目录下,规则见该目录下的规定。


\item tool

tool目录用于存放开发时或构建阶段使用的工具。该目录在发布时 必须(MUST) 被删除。

\item test

test目录用于存放测试用例以及开发阶段的模拟数据。该目录在发布时 必须(MUST) 被删除。

\item doc

doc目录用于存放项目文档。项目文档可能是开发者维护的文档,也可能是通过工具生成的文档。

\item entry

entry目录用于存放项目的页面入口文件,通常是上线后可被直接访问的静态页面。

\item asset目录用于存放用于线上访问的静态资源。

通常构建工具会对src目录和dep目录下的资源进行分析、合并与压缩等,生成到asset目录下。所以该目录尽量避免手工管理。



\end{compactitem}


下面是一个构建工具生成后的asset目录示例:




\begin{lstlisting}[language=bash]
${root}/
    asset/
        js/
            loader.js
            build.js
        css/
            common.css
            img/
        tpl/
            build.tpl.html
        img/
        ...
\end{lstlisting}




RIA项目通常会包含较少的页面入口文件,常见的是main.html,这些文件 可以(SHOULD) 直接放在\$\{root\}目录下。




\begin{lstlisting}[language=bash]
${root}/
    src/
        common/
            conf.js
        card/
        gold/
        message/
    index.html
    main.html
    ......
\end{lstlisting}

多页面项目通常页面入口文件较多, 可以(SHOULD) 统一放在entry目录中,按业务逻辑命名。

\begin{lstlisting}[language=bash]
${root}/
    src/
        common/
            conf.js
        card/
        gold/
        message/
    entry/
        card.html
        gold.html
        message.html
        ......
\end{lstlisting}


项目在发布的时候,构建工具可以页面入口文件为入口进行分析和编译。

RIA项目经过构建工具编译后,目录结构可能如下:

\begin{lstlisting}[language=bash]
output/
    asset/
        js/
        css/
        tpl/
        img/
    index.html
    main.html
\end{lstlisting}

多页面项目经过构建工具编译后,目录结构可能如下:

\begin{lstlisting}[language=bash]
output/
    card/
        asset/
            js/
            css/
            img/
        index.html
    gold/
        asset/
            js/
            css/
            img/
        index.html
\end{lstlisting}


\subsection{Resource}


按资源类型命名的目录称作资源目录。资源目录 不允许(MUST NOT) 直接置于\$\{root\}下。


\begin{compactitem}
\item js目录可用于存放js资源文件(包含可编译成js的coffeescript等语言)。js文件后缀名 必须(MUST) 为.js,coffeescript文件后缀名 必须(MUST) 为.coffee。

js目录内 必须(MUST) 存放js资源文件,但js资源文件不一定(MAY NOT)存放于js目录下:

\begin{compactenum}
\item 对于src目录,js资源文件 不允许(MUST NOT) 存放于js目录下。
\item 对于asset目录,js资源文件 可以(SHOULD) 存放于js目录下,视构建行为决定。
\item 对于其他一级目录内,js资源文件 可以(SHOULD) 不存放于js目录下。
\end{compactenum}

\item css目录可用于存放css资源文件(包含less,sass等动态样式表语言)。css文件后缀名 必须(MUST) 为.css,less文件后缀名 必须(MUST) 为.less。

css目录内 必须(MUST) 存放css资源文件,但css资源文件不一定(MAY NOT)存放于css目录下:

\begin{compactenum}
\item 对于src目录,css资源文件 可以(SHOULD) 存放于业务目录下,也 可以(SHOULD) 存放于css目录下。
\item 对于asset目录,css资源文件 可以(SHOULD) 存放于css目录下,视构建行为决定。
\item 对于其他一级目录内,css资源文件 可以(SHOULD) 不存放于css目录下。
\end{compactenum}

\item img目录可用于存放图片资源文件。包括页面直接引用的图片与css引用图片。常见的图片资源有gif/jpg/png/svg/bmp等。

对于css引用的图片, 必须(MUST) 放在./img目录下,.代表当前css资源所在的目录。

对于页面直接引用的图片:



\begin{compactenum}
\item 被多页面引用的图片 应该(SHOULD) 放在\$\{root\}/src/common/img目录下。
\item 单一页面引用的图片 应该(SHOULD) 放在./img目录下,.代表当前页面所在的目录。
\end{compactenum}

\item tpl目录可用于存放template资源文件。template资源文件后缀名 可以(SHOULD) 为.html或.tpl。

通常,对于RIA系统,template资源文件采用.html后缀使其能够被xhr加载。

\item font目录可用于存放字体资源文件。常见的字体资源有tff/woff/svg等。


\item swf目录可用于存放flash资源文件。flash资源文件 不允许(MUST NOT) 置于img目录中。

\end{compactitem}




\begin{lstlisting}[language=bash]

\end{lstlisting}



\begin{lstlisting}[language=bash]

\end{lstlisting}




\begin{lstlisting}[language=bash]

\end{lstlisting}



\begin{lstlisting}[language=bash]

\end{lstlisting}




\begin{lstlisting}[language=bash]

\end{lstlisting}




\begin{lstlisting}[language=bash]

\end{lstlisting}



\begin{lstlisting}[language=bash]

\end{lstlisting}




\begin{lstlisting}[language=bash]

\end{lstlisting}



\begin{lstlisting}[language=bash]

\end{lstlisting}




\begin{lstlisting}[language=bash]

\end{lstlisting}




\begin{lstlisting}[language=bash]

\end{lstlisting}



\begin{lstlisting}[language=bash]

\end{lstlisting}




\begin{lstlisting}[language=bash]

\end{lstlisting}



\begin{lstlisting}[language=bash]

\end{lstlisting}




\begin{lstlisting}[language=bash]

\end{lstlisting}




\begin{lstlisting}[language=bash]

\end{lstlisting}



\begin{lstlisting}[language=bash]

\end{lstlisting}




\begin{lstlisting}[language=bash]

\end{lstlisting}



\begin{lstlisting}[language=bash]

\end{lstlisting}




\begin{lstlisting}[language=bash]

\end{lstlisting}




\begin{lstlisting}[language=bash]

\end{lstlisting}



\begin{lstlisting}[language=bash]

\end{lstlisting}




\begin{lstlisting}[language=bash]

\end{lstlisting}



\begin{lstlisting}[language=bash]

\end{lstlisting}




\begin{lstlisting}[language=bash]

\end{lstlisting}




\begin{lstlisting}[language=bash]

\end{lstlisting}



\begin{lstlisting}[language=bash]

\end{lstlisting}




\begin{lstlisting}[language=bash]

\end{lstlisting}



\begin{lstlisting}[language=bash]

\end{lstlisting}




\begin{lstlisting}[language=bash]

\end{lstlisting}




\begin{lstlisting}[language=bash]

\end{lstlisting}



\begin{lstlisting}[language=bash]

\end{lstlisting}




\begin{lstlisting}[language=bash]

\end{lstlisting}



\begin{lstlisting}[language=bash]

\end{lstlisting}




\begin{lstlisting}[language=bash]

\end{lstlisting}




\begin{lstlisting}[language=bash]

\end{lstlisting}



\begin{lstlisting}[language=bash]

\end{lstlisting}




\begin{lstlisting}[language=bash]

\end{lstlisting}



\begin{lstlisting}[language=bash]

\end{lstlisting}




\begin{lstlisting}[language=bash]

\end{lstlisting}




\begin{lstlisting}[language=bash]

\end{lstlisting}



\begin{lstlisting}[language=bash]

\end{lstlisting}




\begin{lstlisting}[language=bash]

\end{lstlisting}



\begin{lstlisting}[language=bash]

\end{lstlisting}




\begin{lstlisting}[language=bash]

\end{lstlisting}




\begin{lstlisting}[language=bash]

\end{lstlisting}