\part{Security}


\chapter{Overview}

在OSI(开放系统互联)模型的第七层——应用层向用户的应用程序提供网络服务等。

在实际的网络环境中,仅仅加固网络设备无法防止攻击者通过网络对系统中的Web应用的攻击,而且攻击者可以先入侵目标系统的应用层,然后再伺机进入系统内部。

OWASP开放社区提供了安全编码原则和实践来提高产品的安全性,而且OWASP通过评估攻击向量和安全隐患归纳出的应用安全风险可以涵盖通用的攻击方法,和目标所使用的平台和技术无关,而且OSASP还提供了测试、验证和修复漏洞的指导方案。

\section{Injection}


注入(Injection)是指攻击者通过输入恶意数据来实现在Web服务器中运行任意指令的目的,例如SQL、XML和LDAP注入等。

在应用程序中,通过对用户输入的特定字符进行转义,可以预防恶意数据注入。

\section{XSS}


跨站脚本(Cross-Site Scripting,XSS)是指应用程序在没有对用户输入进行正确验证的情况下,将这些输入直接输出到Web浏览器中,并且在浏览器中执行来实现会话劫持、cookie窃取或者Web站点数据污染等。

在应用程序中,通过对HTML、JavaScript或者CSS输出中不受信任的元字符进行转义,可以预防XSS攻击。


\section{Authentication}

无效的验证和会话管理可以导致用户账户被劫持,或者导致会话token可预测等。

在开发健壮的验证和会话管理程序时,建议使用加密、散列和基于SSL/TLS的安全数据连接。


\section{Reference}

如果应用程序提供其内部对象的直接引用,并且没有进行正确验证,就可能导致攻击者操纵这些引用并访问未经授权的数据。

内部对象可能是用户账户的参数值、文件名或目录,在访问控制检查完成之前限制所有用户可以访问的内部对象,可以确保对相关对象的每一次访问都是经过验证的。

\section{CSRF}

跨站请求伪造(Crosss-Site Request Forgery,CSRF)是指在存在漏洞的Web应用中,强迫经过验证的用户去运行伪造的HTTP请求。

HTTP恶意请求都是在合法的用户会话中被执行的,因此无法检测,在实践中通过在每一个用户会话中都生成一个ie不可预测的token,然后每次发送HTTP请求时都绑定这个token,可以减轻CSRF攻击的危害。

\section{Misconfiguration}

在生产环境中使用默认的额安全配置可能会导致应用程序遭受多种攻击,因此在已部署的应用、Web服务区、数据库服务器、操作系代码库以及所有和应用程序相关的组件中,都应该使用现有的最佳安全配置,并且通过不断地进行软件更新、打补丁和严格执行应用环境中的安全规则来实现安全的应用程序配置。

\section{Cryptography}

如果在应用程序中没有对敏感数据(例如医保信息、信用卡交易、个人信息和认证细节等)使用密码保护机制,那么就可以产生密码泄露等危害。

用户可以使用健壮的加密算法或散列算法来保证敏感数据的安全性。

\section{URL}

如果Web应用程序没有对URL的访问进行权限检查,那么攻击者可能可以访问呢未经授权的网页。

为了防范失败的URL访问权限,需要使用合适的身份证明和授权控制机制来限制对私有URL的访问,同时需要为那些可以访问高敏感性数据的特殊用户和角色开发适当的权限控制策略。

\section{Transport}

低强度的加密算法、无效的安全证书以及不恰当的身份证明控制机制都可能会破坏数据的机密性和完整性,从而使应用数据遭到流量窃听和篡改攻击。

在传输所有敏感网页时使用SSL/TLS协议,并使用权威认证机构办法的合法数字证书,可以解决薄弱的传输层保护问题。

\section{Validation}

如果Web应用程序使用动态参数来将用户重定向或跳转到某个特定的URL上,攻击者就可以通过相同的方法伪造一个恶意的URL来将用户重定向到钓鱼网站或恶意网站上。

未验证的重定向和转发攻击还可以用来将请求转发到本地未经授权的网页上,因此需要验证请求中的参数和发出请求的用户的访问权限来避免非法重定向和转发。






























