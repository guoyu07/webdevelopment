\part{Backup}


在维护数据库和进行文件系统操作(例如dd)时,一定要知道自己在做什么,要做什么。

首先写出步骤,包括连接到哪个数据库,ip是什么,运行什么命令,先做什么,后做什么,出了问题怎么roll back,我知道你都懂,但要写出来,不要相信自己的记忆。

其次在测试环境验证。拿来写好的步骤,在测试环境中跑一遍,一半以上的可能会发现问题,然后再修改步骤,不要直接在产品环境中执行修改。

\begin{compactitem}
\item delete 和 update 前,先查询,用同样的 where 语句 select,至少知道有多少记录会被影响到。
\item drop 和 truncate 之前,检查三遍,连接的是不是正确的数据库。
\end{compactitem}

一次只连接一个DB,不要开几个窗口,有的连测试,有的连产品,或早或晚,你会出错。

最后,备份,备份,备份。



