% !TEX TS-program = xelatex
% !TEX encoding = UTF-8

% This is a simple template for a XeLaTeX document using the "article" class,
% with the fontspec package to easily select fonts.

\documentclass[11pt,adobefonts]{article} % use larger type; default would be 10pt

\usepackage{ctex}


\usepackage{fontspec} % Font selection for XeLaTeX; see fontspec.pdf for documentation
\defaultfontfeatures{Mapping=tex-text} % to support TeX conventions like ``---''
\usepackage{xunicode} % Unicode support for LaTeX character names (accents, European chars, etc)
\usepackage{xltxtra} % Extra customizations for XeLaTeX

\usepackage{paralist}
%\usepackage{xcolor}

\usepackage[colorlinks=true]{hyperref}




\setmainfont{Minion Pro} % set the main body font (\textrm), assumes Charis SIL is installed
%\setsansfont{Deja Vu Sans}
%\setmonofont{Deja Vu Mono}

% other LaTeX packages.....
\usepackage{geometry} % See geometry.pdf to learn the layout options. There are lots.
\geometry{a4paper} % or letterpaper (US) or a5paper or....
%\usepackage[parfill]{parskip} % Activate to begin paragraphs with an empty line rather than an indent

\usepackage{graphicx} % support the \includegraphics command and options

\title{让网站容易被发现}
\author{一叶千鸟}
%\date{} % Activate to display a given date or no date (if empty),
         % otherwise the current date is printed 

\begin{document}
\maketitle

%\section{}

%\subsection{}

第一次见到《Web标准和SEO应用实践》是在我们UCD书友会《设计沟通十器》新书发布现场,机械工业出版社的朋友还带了很多相关新书过来,当时本书中文译者李清也在场。

Peter Morville在04年总结的User Experience Design中就涉及到Findable,我大概能想到些实际应用。但当真正去查找web-based findable相关技术资料做总结时,发现少的可怜,唯一亮点就是Aarron Walter写的这本《Building Findable Website》。

其实本书中文名翻译仅仅用了原版书名的副标题,web standards, seo, and beyond.我想只是作者用来对Findable的补充,现在标题可能会给书留下名不符其实的传播隐患。准确的Building Findable Website可以叫《构建良好可发现性网站》,或者通俗点用本网志标题也可以。

我的理解,SEF(Search Engine Friendly)与实现用户友好的可发现性、可访问性有交集,其中已包括W3C标准。只不过SEF在搜索引擎角度考虑问题,基本与书中理念相吻合。此书主要贡献,在于分别从标记语言、服务器、内容、导航、搜索等角度,做了较深入总结,所举blog例子也比较实用。涉及范畴基本可参考最后总结,有删减: 

\section{Level 1}


\begin{compactenum}
\item 创建填补空白,与用户相关的精美原创内容;
\item 研究目标关键字,并放置在标记内的战略位置(关键字密度不应超过7个百分点);
\item 使用语义的、标准兼容的代码;
\item 满足易访问性指南,以确保内容对搜索引擎是易读的;
\item 建立可预测、搜索引擎友好的URL;
\item 如果可能,建立站内链接,并在其他网站上宣传你的网站;
\item 发布robots.txt和sitemap.xml,然后将sitemap.xml通知给各大搜索引擎;
\item 创建定制的404页,使用户能够返回;
\item 建立HTML网站地图页面,来帮助用户和搜索引擎浏览网站;
\item 确保JavaScript以及Flash内容不会阻止搜索引擎索引;
\item 坚持分析成功和失败的流量;
\end{compactenum}

\section{Level 2}

\begin{compactenum}
\item 给网站添加本地搜索引擎;
\item 用微格式使得如活动和联络方式等内容易于携带;
\item 创建博客,有新内容通知主要的ping服务;
\item 可能的话,用RSS聚合内容;
\item 促进内容的病毒式交流;
\item 优化网站性能,使其有效的索引;
\end{compactenum}

\section{Level 3}

\begin{compactenum}
\item 建立和利用邮件列表;
\item 考虑通过Google的AdWords活动来吸引即时的流量;
\item 通过线下媒体广告来推广你的网站;
\end{compactenum}

相对来说,我喜欢这样的行文,不是那种宏篇巨著论述一句话的风格。我认为读技术类书籍最好状态是互相切磋,拿自己总结与作者总结对比取长补短。而不是一穷二白的“接受”所述内容,别人说什么你就同意什么,没有自己见解,也没有自己思考。

有些书的内容其实是模棱两可,比如讲概念肯定都希望自己创新的概念被拔高、受重视,但实际上这个概念仅仅是系统知识框架中的一小点而已。如果读者没有实践和研究积累去读书,很容易被片面知识蒙蔽更宽的视野。比如本书也不免俗的在开头就画了大大的概念图,来传达Findable有多么重要和高深,懂的人不看也自然懂,不懂的人看了更不懂。

客观评价,如果聚焦在Findable概念范畴之内是本难得的好书,前提是读者必须深刻认识什么叫Findable。豆瓣上就有被标题误导的朋友说“如果是为了学习seo的话别买这本书”,虽然没有涉及任何SE的Optimization内容,但所总结SE的Friendly都具有良好可操作性,何况很多同行根本连SEO和SEF都分不清楚。

\textsl{Building Findable Websites: Web Standards, SEO, and Beyond}

\url{http://buildingfindablewebsites.com}




\end{document}
